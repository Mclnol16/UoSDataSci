\documentclass[]{article}
\usepackage{lmodern}
\usepackage{amssymb,amsmath}
\usepackage{ifxetex,ifluatex}
\usepackage{fixltx2e} % provides \textsubscript
\ifnum 0\ifxetex 1\fi\ifluatex 1\fi=0 % if pdftex
  \usepackage[T1]{fontenc}
  \usepackage[utf8]{inputenc}
\else % if luatex or xelatex
  \ifxetex
    \usepackage{mathspec}
  \else
    \usepackage{fontspec}
  \fi
  \defaultfontfeatures{Ligatures=TeX,Scale=MatchLowercase}
\fi
% use upquote if available, for straight quotes in verbatim environments
\IfFileExists{upquote.sty}{\usepackage{upquote}}{}
% use microtype if available
\IfFileExists{microtype.sty}{%
\usepackage{microtype}
\UseMicrotypeSet[protrusion]{basicmath} % disable protrusion for tt fonts
}{}
\usepackage[margin=1in]{geometry}
\usepackage{hyperref}
\hypersetup{unicode=true,
            pdftitle={R Notebook on Baseball Salaries Data},
            pdfborder={0 0 0},
            breaklinks=true}
\urlstyle{same}  % don't use monospace font for urls
\usepackage{color}
\usepackage{fancyvrb}
\newcommand{\VerbBar}{|}
\newcommand{\VERB}{\Verb[commandchars=\\\{\}]}
\DefineVerbatimEnvironment{Highlighting}{Verbatim}{commandchars=\\\{\}}
% Add ',fontsize=\small' for more characters per line
\usepackage{framed}
\definecolor{shadecolor}{RGB}{248,248,248}
\newenvironment{Shaded}{\begin{snugshade}}{\end{snugshade}}
\newcommand{\KeywordTok}[1]{\textcolor[rgb]{0.13,0.29,0.53}{\textbf{#1}}}
\newcommand{\DataTypeTok}[1]{\textcolor[rgb]{0.13,0.29,0.53}{#1}}
\newcommand{\DecValTok}[1]{\textcolor[rgb]{0.00,0.00,0.81}{#1}}
\newcommand{\BaseNTok}[1]{\textcolor[rgb]{0.00,0.00,0.81}{#1}}
\newcommand{\FloatTok}[1]{\textcolor[rgb]{0.00,0.00,0.81}{#1}}
\newcommand{\ConstantTok}[1]{\textcolor[rgb]{0.00,0.00,0.00}{#1}}
\newcommand{\CharTok}[1]{\textcolor[rgb]{0.31,0.60,0.02}{#1}}
\newcommand{\SpecialCharTok}[1]{\textcolor[rgb]{0.00,0.00,0.00}{#1}}
\newcommand{\StringTok}[1]{\textcolor[rgb]{0.31,0.60,0.02}{#1}}
\newcommand{\VerbatimStringTok}[1]{\textcolor[rgb]{0.31,0.60,0.02}{#1}}
\newcommand{\SpecialStringTok}[1]{\textcolor[rgb]{0.31,0.60,0.02}{#1}}
\newcommand{\ImportTok}[1]{#1}
\newcommand{\CommentTok}[1]{\textcolor[rgb]{0.56,0.35,0.01}{\textit{#1}}}
\newcommand{\DocumentationTok}[1]{\textcolor[rgb]{0.56,0.35,0.01}{\textbf{\textit{#1}}}}
\newcommand{\AnnotationTok}[1]{\textcolor[rgb]{0.56,0.35,0.01}{\textbf{\textit{#1}}}}
\newcommand{\CommentVarTok}[1]{\textcolor[rgb]{0.56,0.35,0.01}{\textbf{\textit{#1}}}}
\newcommand{\OtherTok}[1]{\textcolor[rgb]{0.56,0.35,0.01}{#1}}
\newcommand{\FunctionTok}[1]{\textcolor[rgb]{0.00,0.00,0.00}{#1}}
\newcommand{\VariableTok}[1]{\textcolor[rgb]{0.00,0.00,0.00}{#1}}
\newcommand{\ControlFlowTok}[1]{\textcolor[rgb]{0.13,0.29,0.53}{\textbf{#1}}}
\newcommand{\OperatorTok}[1]{\textcolor[rgb]{0.81,0.36,0.00}{\textbf{#1}}}
\newcommand{\BuiltInTok}[1]{#1}
\newcommand{\ExtensionTok}[1]{#1}
\newcommand{\PreprocessorTok}[1]{\textcolor[rgb]{0.56,0.35,0.01}{\textit{#1}}}
\newcommand{\AttributeTok}[1]{\textcolor[rgb]{0.77,0.63,0.00}{#1}}
\newcommand{\RegionMarkerTok}[1]{#1}
\newcommand{\InformationTok}[1]{\textcolor[rgb]{0.56,0.35,0.01}{\textbf{\textit{#1}}}}
\newcommand{\WarningTok}[1]{\textcolor[rgb]{0.56,0.35,0.01}{\textbf{\textit{#1}}}}
\newcommand{\AlertTok}[1]{\textcolor[rgb]{0.94,0.16,0.16}{#1}}
\newcommand{\ErrorTok}[1]{\textcolor[rgb]{0.64,0.00,0.00}{\textbf{#1}}}
\newcommand{\NormalTok}[1]{#1}
\usepackage{graphicx,grffile}
\makeatletter
\def\maxwidth{\ifdim\Gin@nat@width>\linewidth\linewidth\else\Gin@nat@width\fi}
\def\maxheight{\ifdim\Gin@nat@height>\textheight\textheight\else\Gin@nat@height\fi}
\makeatother
% Scale images if necessary, so that they will not overflow the page
% margins by default, and it is still possible to overwrite the defaults
% using explicit options in \includegraphics[width, height, ...]{}
\setkeys{Gin}{width=\maxwidth,height=\maxheight,keepaspectratio}
\IfFileExists{parskip.sty}{%
\usepackage{parskip}
}{% else
\setlength{\parindent}{0pt}
\setlength{\parskip}{6pt plus 2pt minus 1pt}
}
\setlength{\emergencystretch}{3em}  % prevent overfull lines
\providecommand{\tightlist}{%
  \setlength{\itemsep}{0pt}\setlength{\parskip}{0pt}}
\setcounter{secnumdepth}{0}
% Redefines (sub)paragraphs to behave more like sections
\ifx\paragraph\undefined\else
\let\oldparagraph\paragraph
\renewcommand{\paragraph}[1]{\oldparagraph{#1}\mbox{}}
\fi
\ifx\subparagraph\undefined\else
\let\oldsubparagraph\subparagraph
\renewcommand{\subparagraph}[1]{\oldsubparagraph{#1}\mbox{}}
\fi

%%% Use protect on footnotes to avoid problems with footnotes in titles
\let\rmarkdownfootnote\footnote%
\def\footnote{\protect\rmarkdownfootnote}

%%% Change title format to be more compact
\usepackage{titling}

% Create subtitle command for use in maketitle
\newcommand{\subtitle}[1]{
  \posttitle{
    \begin{center}\large#1\end{center}
    }
}

\setlength{\droptitle}{-2em}

  \title{R Notebook on Baseball Salaries Data}
    \pretitle{\vspace{\droptitle}\centering\huge}
  \posttitle{\par}
    \author{}
    \preauthor{}\postauthor{}
    \date{}
    \predate{}\postdate{}
  

\begin{document}
\maketitle

\section{Introduction and Data
Summary}\label{introduction-and-data-summary}

In this notebook we will carry out some analyses on the dataset
Salaries.csv.

We begin ny loading the libraries we will use and importing the
appropriate libraries. Namely, we load the tidyverse and reshape2
packages.

\begin{Shaded}
\begin{Highlighting}[]
\KeywordTok{library}\NormalTok{(tidyverse)}
\end{Highlighting}
\end{Shaded}

\begin{verbatim}
## -- Attaching packages ------------------------------------------------------------------- tidyverse 1.2.1 --
\end{verbatim}

\begin{verbatim}
## √ ggplot2 3.0.0     √ purrr   0.2.5
## √ tibble  1.4.2     √ dplyr   0.7.6
## √ tidyr   0.8.1     √ stringr 1.3.1
## √ readr   1.1.1     √ forcats 0.3.0
\end{verbatim}

\begin{verbatim}
## Warning: package 'ggplot2' was built under R version 3.4.4
\end{verbatim}

\begin{verbatim}
## Warning: package 'tidyr' was built under R version 3.4.4
\end{verbatim}

\begin{verbatim}
## Warning: package 'purrr' was built under R version 3.4.4
\end{verbatim}

\begin{verbatim}
## Warning: package 'dplyr' was built under R version 3.4.4
\end{verbatim}

\begin{verbatim}
## Warning: package 'stringr' was built under R version 3.4.4
\end{verbatim}

\begin{verbatim}
## -- Conflicts ---------------------------------------------------------------------- tidyverse_conflicts() --
## x dplyr::filter() masks stats::filter()
## x dplyr::lag()    masks stats::lag()
\end{verbatim}

\begin{Shaded}
\begin{Highlighting}[]
\KeywordTok{library}\NormalTok{(reshape2)}
\end{Highlighting}
\end{Shaded}

\begin{verbatim}
## 
## Attaching package: 'reshape2'
\end{verbatim}

\begin{verbatim}
## The following object is masked from 'package:tidyr':
## 
##     smiths
\end{verbatim}

Next, import the data.

\begin{Shaded}
\begin{Highlighting}[]
\NormalTok{salary_df <-}\StringTok{ }\KeywordTok{read_csv}\NormalTok{(}\StringTok{"Data/Salaries.csv"}\NormalTok{)}
\end{Highlighting}
\end{Shaded}

\begin{verbatim}
## Parsed with column specification:
## cols(
##   yearID = col_integer(),
##   teamID = col_character(),
##   lgID = col_character(),
##   playerID = col_character(),
##   salary = col_integer()
## )
\end{verbatim}

We should determine the size and column names of the dataframe.

\begin{Shaded}
\begin{Highlighting}[]
\KeywordTok{dim}\NormalTok{(salary_df)}
\end{Highlighting}
\end{Shaded}

\begin{verbatim}
## [1] 24758     5
\end{verbatim}

\begin{Shaded}
\begin{Highlighting}[]
\KeywordTok{names}\NormalTok{(salary_df)}
\end{Highlighting}
\end{Shaded}

\begin{verbatim}
## [1] "yearID"   "teamID"   "lgID"     "playerID" "salary"
\end{verbatim}

Then examine the first few and last few rows of the data.

\begin{Shaded}
\begin{Highlighting}[]
\KeywordTok{head}\NormalTok{(salary_df)}
\end{Highlighting}
\end{Shaded}

\begin{verbatim}
## # A tibble: 6 x 5
##   yearID teamID lgID  playerID  salary
##    <int> <chr>  <chr> <chr>      <int>
## 1   1985 ATL    NL    barkele01 870000
## 2   1985 ATL    NL    bedrost01 550000
## 3   1985 ATL    NL    benedbr01 545000
## 4   1985 ATL    NL    campri01  633333
## 5   1985 ATL    NL    ceronri01 625000
## 6   1985 ATL    NL    chambch01 800000
\end{verbatim}

\begin{Shaded}
\begin{Highlighting}[]
\KeywordTok{tail}\NormalTok{(salary_df)}
\end{Highlighting}
\end{Shaded}

\begin{verbatim}
## # A tibble: 6 x 5
##   yearID teamID lgID  playerID    salary
##    <int> <chr>  <chr> <chr>        <int>
## 1   2014 WAS    NL    stammcr01  1375000
## 2   2014 WAS    NL    storedr01  3450000
## 3   2014 WAS    NL    strasst01  3975000
## 4   2014 WAS    NL    werthja01 20000000
## 5   2014 WAS    NL    zimmejo02  7500000
## 6   2014 WAS    NL    zimmery01 14000000
\end{verbatim}

We can quickly obtain some summary statistics.

\begin{Shaded}
\begin{Highlighting}[]
\KeywordTok{summary}\NormalTok{(salary_df)}
\end{Highlighting}
\end{Shaded}

\begin{verbatim}
##      yearID        teamID              lgID             playerID        
##  Min.   :1985   Length:24758       Length:24758       Length:24758      
##  1st Qu.:1993   Class :character   Class :character   Class :character  
##  Median :2000   Mode  :character   Mode  :character   Mode  :character  
##  Mean   :2000                                                           
##  3rd Qu.:2007                                                           
##  Max.   :2014                                                           
##      salary        
##  Min.   :       0  
##  1st Qu.:  260000  
##  Median :  525000  
##  Mean   : 1932905  
##  3rd Qu.: 2199643  
##  Max.   :33000000
\end{verbatim}

This tells us some useful information. First, we see the class of the
five variables. Second, there is no missing data becuase otherwise this
would have been indicated by the output from the summary function. In
addition, we get the five mean and quantiles for the two numeric
variables yearID and salary.

What information does this dataset seem to contain?

\section{Some Initial Plots}\label{some-initial-plots}

\begin{Shaded}
\begin{Highlighting}[]
\NormalTok{salary_df }\OperatorTok\StringTok{ }\KeywordTok{ggplot}\NormalTok{() }\OperatorTok{+}\StringTok{ }\KeywordTok{geom_bar}\NormalTok{(}\DataTypeTok{mapping =} \KeywordTok{aes}\NormalTok{(}\DataTypeTok{x=}\NormalTok{teamID)) }\OperatorTok{+}\StringTok{ }\KeywordTok{coord_flip}\NormalTok{()}
\end{Highlighting}
\end{Shaded}

\includegraphics{baseballSalariesNotebook_files/figure-latex/unnamed-chunk-7-1.pdf}

This shows how many observations there are for each of the 37 different
teams. Notice that one way to determine how many distinct teams there
are is to use the following command.

\begin{Shaded}
\begin{Highlighting}[]
\NormalTok{salary_df}\OperatorTok{$}\NormalTok{teamID }\OperatorTok\StringTok{ }\KeywordTok{unique}\NormalTok{() }\OperatorTok\StringTok{ }\KeywordTok{length}\NormalTok{()}
\end{Highlighting}
\end{Shaded}

\begin{verbatim}
## [1] 37
\end{verbatim}

\section{Examine Salaries by Year}\label{examine-salaries-by-year}

\begin{Shaded}
\begin{Highlighting}[]
\NormalTok{avg_sal_by_year <-}\StringTok{ }\NormalTok{salary_df }\OperatorTok\StringTok{ }\KeywordTok{group_by}\NormalTok{(yearID) }\OperatorTok\StringTok{ }\KeywordTok{summarize}\NormalTok{(}\DataTypeTok{avg_sal =} \KeywordTok{mean}\NormalTok{(salary))}
\end{Highlighting}
\end{Shaded}

Let's take a look.

\begin{Shaded}
\begin{Highlighting}[]
\KeywordTok{head}\NormalTok{(avg_sal_by_year)}
\end{Highlighting}
\end{Shaded}

\begin{verbatim}
## # A tibble: 6 x 2
##   yearID avg_sal
##    <int>   <dbl>
## 1   1985 476299.
## 2   1986 417147.
## 3   1987 434729.
## 4   1988 453171.
## 5   1989 506323.
## 6   1990 511974.
\end{verbatim}

We can kind of check this. For example keep only data with year 1985 and
compute mean

\begin{Shaded}
\begin{Highlighting}[]
\NormalTok{salary_df }\OperatorTok\StringTok{ }\KeywordTok{filter}\NormalTok{(yearID }\OperatorTok{==}\StringTok{ }\DecValTok{1985}\NormalTok{) }\OperatorTok\StringTok{ }\KeywordTok{select}\NormalTok{(salary) }\OperatorTok\StringTok{ }\KeywordTok{unlist}\NormalTok{() }\OperatorTok\StringTok{ }\KeywordTok{mean}\NormalTok{()}
\end{Highlighting}
\end{Shaded}

\begin{verbatim}
## Warning: package 'bindrcpp' was built under R version 3.4.4
\end{verbatim}

\begin{verbatim}
## [1] 476299.4
\end{verbatim}

This agrees with what we saw in the first row for avg\_sal\_by\_year.

Now let's plot the mean salary over time.

\begin{Shaded}
\begin{Highlighting}[]
\NormalTok{avg_sal_by_year }\OperatorTok\StringTok{ }\KeywordTok{ggplot}\NormalTok{(}\DataTypeTok{mapping =} \KeywordTok{aes}\NormalTok{(}\DataTypeTok{x=}\NormalTok{yearID,}\DataTypeTok{y=}\NormalTok{avg_sal)) }\OperatorTok{+}\StringTok{ }\KeywordTok{geom_line}\NormalTok{()}
\end{Highlighting}
\end{Shaded}

\includegraphics{baseballSalariesNotebook_files/figure-latex/unnamed-chunk-12-1.pdf}

What does this plot suggest?

\section{A Deeper Analysis}\label{a-deeper-analysis}

We should be careful regarding outliers so let's look at the median too.

\begin{Shaded}
\begin{Highlighting}[]
\NormalTok{sal_stats_by_year <-}\StringTok{ }\NormalTok{salary_df }\OperatorTok\StringTok{ }
\StringTok{  }\KeywordTok{group_by}\NormalTok{(yearID) }\OperatorTok\StringTok{ }
\StringTok{  }\KeywordTok{summarize}\NormalTok{(}\DataTypeTok{avg_sal =} \KeywordTok{mean}\NormalTok{(salary), }\DataTypeTok{med_sal =} \KeywordTok{median}\NormalTok{(salary))}
\end{Highlighting}
\end{Shaded}

We can plot both mean and median salaries over time but first we have to
convert data to long format.

\begin{Shaded}
\begin{Highlighting}[]
\NormalTok{sal_stats_by_year }\OperatorTok\StringTok{ }
\StringTok{  }\KeywordTok{melt}\NormalTok{(}\DataTypeTok{id=}\StringTok{"yearID"}\NormalTok{) }\OperatorTok
\StringTok{  }\KeywordTok{ggplot}\NormalTok{(}\DataTypeTok{mapping=}\KeywordTok{aes}\NormalTok{(}\DataTypeTok{x=}\NormalTok{yearID, }\DataTypeTok{y=}\NormalTok{value, }\DataTypeTok{colour=}\NormalTok{variable)) }\OperatorTok{+}
\StringTok{  }\KeywordTok{geom_line}\NormalTok{()}
\end{Highlighting}
\end{Shaded}

\includegraphics{baseballSalariesNotebook_files/figure-latex/unnamed-chunk-14-1.pdf}

What can you conclude?

It is easy to look at the salary distributions by team.

\begin{Shaded}
\begin{Highlighting}[]
\NormalTok{salary_df }\OperatorTok\StringTok{ }\KeywordTok{ggplot}\NormalTok{(}\DataTypeTok{mapping =} \KeywordTok{aes}\NormalTok{(}\DataTypeTok{x=}\NormalTok{salary)) }\OperatorTok{+}\StringTok{ }\KeywordTok{geom_histogram}\NormalTok{() }\OperatorTok{+}\StringTok{ }\KeywordTok{facet_wrap}\NormalTok{(}\OperatorTok{~}\StringTok{ }\NormalTok{teamID)}
\end{Highlighting}
\end{Shaded}

\begin{verbatim}
## `stat_bin()` using `bins = 30`. Pick better value with `binwidth`.
\end{verbatim}

\includegraphics{baseballSalariesNotebook_files/figure-latex/unnamed-chunk-15-1.pdf}

Let's write a function that plots the salary distribution for input year
and team

\begin{Shaded}
\begin{Highlighting}[]
\NormalTok{team_year_sal <-}\StringTok{ }\ControlFlowTok{function}\NormalTok{(yearID_val,teamID_val)\{}
\NormalTok{  salary_df }\OperatorTok\StringTok{ }\KeywordTok{filter}\NormalTok{(teamID}\OperatorTok{==}\NormalTok{teamID_val,yearID }\OperatorTok{==}\StringTok{ }\NormalTok{yearID_val) }\OperatorTok
\StringTok{    }\KeywordTok{ggplot}\NormalTok{(}\DataTypeTok{mapping =} \KeywordTok{aes}\NormalTok{(}\DataTypeTok{x=}\NormalTok{salary)) }\OperatorTok{+}\StringTok{ }
\StringTok{    }\KeywordTok{geom_histogram}\NormalTok{() }\OperatorTok{+}\StringTok{ }
\StringTok{    }\KeywordTok{ggtitle}\NormalTok{(}\KeywordTok{paste}\NormalTok{(}\KeywordTok{as.character}\NormalTok{(yearID_val),teamID_val,}\DataTypeTok{sep=}\StringTok{" "}\NormalTok{))}
\NormalTok{\}}
\end{Highlighting}
\end{Shaded}

Example using the function.

\begin{Shaded}
\begin{Highlighting}[]
\KeywordTok{team_year_sal}\NormalTok{(}\DecValTok{2014}\NormalTok{,}\StringTok{"TEX"}\NormalTok{)}
\end{Highlighting}
\end{Shaded}

\begin{verbatim}
## `stat_bin()` using `bins = 30`. Pick better value with `binwidth`.
\end{verbatim}

\includegraphics{baseballSalariesNotebook_files/figure-latex/unnamed-chunk-17-1.pdf}

Let's do something similar but produce a boxplot instead.

\begin{Shaded}
\begin{Highlighting}[]
\NormalTok{team_year_sal_box <-}\StringTok{ }\ControlFlowTok{function}\NormalTok{(yearID_val,teamID_val)\{}
\NormalTok{  salary_df }\OperatorTok\StringTok{ }\KeywordTok{filter}\NormalTok{(teamID}\OperatorTok{==}\NormalTok{teamID_val,yearID }\OperatorTok{==}\StringTok{ }\NormalTok{yearID_val) }\OperatorTok
\StringTok{    }\KeywordTok{ggplot}\NormalTok{(}\DataTypeTok{mapping =} \KeywordTok{aes}\NormalTok{(}\DataTypeTok{y=}\NormalTok{salary)) }\OperatorTok{+}\StringTok{ }
\StringTok{    }\KeywordTok{geom_boxplot}\NormalTok{() }\OperatorTok{+}\StringTok{ }
\StringTok{    }\KeywordTok{ggtitle}\NormalTok{(}\KeywordTok{paste}\NormalTok{(}\KeywordTok{as.character}\NormalTok{(yearID_val),teamID_val,}\DataTypeTok{sep=}\StringTok{" "}\NormalTok{))}
\NormalTok{\}}
\end{Highlighting}
\end{Shaded}

Example using the function

\begin{Shaded}
\begin{Highlighting}[]
\KeywordTok{team_year_sal_box}\NormalTok{(}\DecValTok{2014}\NormalTok{,}\StringTok{"TEX"}\NormalTok{)}
\end{Highlighting}
\end{Shaded}

\includegraphics{baseballSalariesNotebook_files/figure-latex/unnamed-chunk-19-1.pdf}

Perhaps we can see the boxplots across all teams for a given year

\begin{Shaded}
\begin{Highlighting}[]
\NormalTok{year_sal_box <-}\StringTok{ }\ControlFlowTok{function}\NormalTok{(yearID_val)\{}
\NormalTok{  salary_df }\OperatorTok\StringTok{ }\KeywordTok{filter}\NormalTok{(yearID }\OperatorTok{==}\StringTok{ }\NormalTok{yearID_val) }\OperatorTok
\StringTok{    }\KeywordTok{ggplot}\NormalTok{(}\DataTypeTok{mapping =} \KeywordTok{aes}\NormalTok{(}\DataTypeTok{x=}\NormalTok{teamID,}\DataTypeTok{y=}\NormalTok{salary)) }\OperatorTok{+}\StringTok{ }
\StringTok{    }\KeywordTok{geom_boxplot}\NormalTok{() }\OperatorTok{+}\StringTok{ }
\StringTok{    }\KeywordTok{coord_flip}\NormalTok{() }\OperatorTok{+}\StringTok{ }
\StringTok{    }\KeywordTok{ggtitle}\NormalTok{(}\KeywordTok{as.character}\NormalTok{(yearID_val))}
\NormalTok{\}}
\end{Highlighting}
\end{Shaded}

Example using the function

\begin{Shaded}
\begin{Highlighting}[]
\KeywordTok{year_sal_box}\NormalTok{(}\DecValTok{2014}\NormalTok{)}
\end{Highlighting}
\end{Shaded}

\includegraphics{baseballSalariesNotebook_files/figure-latex/unnamed-chunk-21-1.pdf}

What can we see based on this?

Let's find the total salary amount per team per year.

\begin{Shaded}
\begin{Highlighting}[]
\NormalTok{tot_sal <-}\StringTok{ }\NormalTok{salary_df }\OperatorTok\StringTok{ }\KeywordTok{group_by}\NormalTok{(yearID,teamID) }\OperatorTok\StringTok{ }\KeywordTok{summarise}\NormalTok{(}\DataTypeTok{total_salary=}\KeywordTok{sum}\NormalTok{(salary))}
\end{Highlighting}
\end{Shaded}

Take a look at result.

\begin{Shaded}
\begin{Highlighting}[]
\KeywordTok{head}\NormalTok{(tot_sal)}
\end{Highlighting}
\end{Shaded}

\begin{verbatim}
## # A tibble: 6 x 3
## # Groups:   yearID [1]
##   yearID teamID total_salary
##    <int> <chr>         <int>
## 1   1985 ATL        14807000
## 2   1985 BAL        11560712
## 3   1985 BOS        10897560
## 4   1985 CAL        14427894
## 5   1985 CHA         9846178
## 6   1985 CHN        12702917
\end{verbatim}

Let's determine for each year which team spent the most:

\begin{Shaded}
\begin{Highlighting}[]
\CommentTok{# First we find the max total salaries by year}
\NormalTok{tot_max <-}\StringTok{ }\NormalTok{tot_sal }\OperatorTok\StringTok{ }\KeywordTok{group_by}\NormalTok{(yearID) }\OperatorTok\StringTok{ }\KeywordTok{summarise}\NormalTok{(}\DataTypeTok{max_sal=}\KeywordTok{max}\NormalTok{(total_salary))}
\CommentTok{# Then convert the max total salary values into a vector}
\NormalTok{max_vals <-}\StringTok{ }\NormalTok{tot_max}\OperatorTok{$}\NormalTok{max_sal }\OperatorTok\StringTok{ }\KeywordTok{unlist}\NormalTok{()}
\CommentTok{# Then pull out the rows where the max total salaries occur}
\NormalTok{high_price_teams <-}\StringTok{ }\NormalTok{tot_sal }\OperatorTok\StringTok{ }\KeywordTok{filter}\NormalTok{(total_salary }\OperatorTok\StringTok{ }\NormalTok{max_vals)}
\end{Highlighting}
\end{Shaded}

Let's look.

\begin{Shaded}
\begin{Highlighting}[]
\KeywordTok{head}\NormalTok{(high_price_teams)}
\end{Highlighting}
\end{Shaded}

\begin{verbatim}
## # A tibble: 6 x 3
## # Groups:   yearID [6]
##   yearID teamID total_salary
##    <int> <chr>         <int>
## 1   1985 ATL        14807000
## 2   1986 NYA        18494253
## 3   1987 NYA        17099714
## 4   1988 NYA        19441152
## 5   1989 LAN        21071562
## 6   1990 KCA        23361084
\end{verbatim}

What does the previous result tell us?

There's only 30 rows and 3 columns so we can look at the entire result.

\begin{Shaded}
\begin{Highlighting}[]
\KeywordTok{View}\NormalTok{(high_price_teams)}
\end{Highlighting}
\end{Shaded}

\begin{verbatim}
## Warning: running command ''/usr/bin/otool' -L '/Library/Frameworks/
## R.framework/Resources/modules/R_de.so'' had status 1
\end{verbatim}

Let's plot the maximum total salaries over time.

\begin{Shaded}
\begin{Highlighting}[]
\NormalTok{high_price_teams }\OperatorTok\StringTok{ }
\StringTok{  }\KeywordTok{ggplot}\NormalTok{() }\OperatorTok{+}\StringTok{ }
\StringTok{  }\KeywordTok{geom_point}\NormalTok{(}\DataTypeTok{mapping =} \KeywordTok{aes}\NormalTok{(}\DataTypeTok{x=}\NormalTok{yearID,}\DataTypeTok{y=}\NormalTok{total_salary,}\DataTypeTok{color=}\NormalTok{teamID)) }
\end{Highlighting}
\end{Shaded}

\includegraphics{baseballSalariesNotebook_files/figure-latex/unnamed-chunk-27-1.pdf}

It could be useful to fit a curve through the data.

\begin{Shaded}
\begin{Highlighting}[]
\NormalTok{high_price_teams }\OperatorTok\StringTok{ }
\StringTok{  }\KeywordTok{ggplot}\NormalTok{() }\OperatorTok{+}\StringTok{ }
\StringTok{  }\KeywordTok{geom_point}\NormalTok{(}\DataTypeTok{mapping =} \KeywordTok{aes}\NormalTok{(}\DataTypeTok{x=}\NormalTok{yearID,}\DataTypeTok{y=}\NormalTok{total_salary,}\DataTypeTok{color=}\NormalTok{teamID)) }\OperatorTok{+}\StringTok{ }
\StringTok{  }\KeywordTok{geom_smooth}\NormalTok{(}\DataTypeTok{mapping =} \KeywordTok{aes}\NormalTok{(}\DataTypeTok{y=}\NormalTok{total_salary, }\DataTypeTok{x=}\NormalTok{yearID))}
\end{Highlighting}
\end{Shaded}

\begin{verbatim}
## `geom_smooth()` using method = 'loess' and formula 'y ~ x'
\end{verbatim}

\includegraphics{baseballSalariesNotebook_files/figure-latex/unnamed-chunk-28-1.pdf}

It would be interesting to know the playoff and world series results for
these high salary teams.


\end{document}
