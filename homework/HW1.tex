\documentclass[12pt]{article}
\textwidth=7in
\textheight=9.5in
\topmargin=-1in
\headheight=0in
\headsep=.5in
\hoffset  -.85in

\pagestyle{empty}

\usepackage{amsmath,amssymb,amsfonts}
\usepackage{url}

\renewcommand{\thefootnote}{\fnsymbol{footnote}}
\begin{document}

\begin{center}
{\bf Introduction to Data Science \\ Homework 1: Due Friday August 31 at 2:00pm}
\end{center}

\setlength{\unitlength}{1in}

\begin{picture}(6,.1)
\put(0,0) {\line(1,0){6.5}}
\end{picture}

\renewcommand{\arraystretch}{2}

\vskip.25in

\noindent{\bf  {\Large Exercises:} }

\vskip.25in
  \begin{enumerate}
    \item Read Chapter 1: What is Data Science? from The Data Science Design Manual
    by Skiena, then respond to the questions listed below. Be sure to fully explain or justify your answer. You may also be interested to
    watch the corresponding video lecture here \url{http://www3.cs.stonybrook.edu/~skiena/data-manual/lectures/}.
    \begin{enumerate}
      \item What is the relationship between data science and the discplines of
      computer science, mathematics, and statistics?
      \item What are some substantive application domains where data science is relevant? Explain your answer.
      \item Provide an argument for or against the following statement: ``Data science is an indepedent discipline and not
      simply an application or a rehashing of traditional ideas from established fields such as statistics.''
      \item Give a distinction between hypothesis-driven science and data-driven science.
      \item Provide several examples of the types of general questions a data scientist may ask.
      \item Find a source of data different from  but in the same vein as those
        described in sections 1.2.1 - 1.2.4 from the reading. Describe this data source and list examples of
        questions that you could use your data source to answer. Use sections 1.2.1 - 1.2.4 as templates.
      \item List the ``taxonomy of data.'' Explain the following terms in your own words and provide an example to illustrate the concept:
      \begin{itemize}
        \item quantitative data
        \item categorical data
      \end{itemize}
      \item Provide an example of a classification problem different from the examples given in the reading.
      \item Provide an example of a regression problem different from the examples given in the reading.
      \item What is Kaggle and what are Kaggle challenges?
    \end{enumerate}
    \item Visit \url{http://data.gov}, and identify five data sets that sound interesting to you. For each write a brief description
    and propose three interesting things you might do with them.
    \item You would like to conduct an experiment to establish whether your friends prefer the taste of Coke or Dr.\ Pepper. Briefly outline
    a design for such a study.
  \end{enumerate}
\end{document}
