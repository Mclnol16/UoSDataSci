\documentclass[12pt]{article}
\textwidth=7in
\textheight=9.5in
\topmargin=-1in
\headheight=0in
\headsep=.5in
\hoffset  -.85in

\pagestyle{empty}

\usepackage{amsmath,amssymb,amsfonts}
\usepackage{url}

\renewcommand{\thefootnote}{\fnsymbol{footnote}}
\begin{document}

\begin{center}
{\bf Introduction to Data Science \\ Homework 4: Due Wednesday September 26 at 2:00pm}
\end{center}

\setlength{\unitlength}{1in}

\begin{picture}(6,.1)
\put(0,0) {\line(1,0){6.5}}
\end{picture}

\renewcommand{\arraystretch}{2}

\vskip.25in

\noindent{\bf  {\Large Exercises:} }

\vskip.25in
  \begin{enumerate}
    \item   Go through section 9 Functions of the R Programming course in the {\tt swirl} package, then answer the following questions:
    \begin{enumerate}
      \item   What is a function? 
      \item What does the {\tt Sys.Data()} function do? How many input arguments are required?
      \item What are the two ``slogans'' for R stated by John Chambers? 
      \item How do you see the source code for an R function? 
      \item Why would having default arguments by useful?
      \item What does the {\tt args} function do? Give an example of its use. 
      \item Explain why one might want to pass a function as an argument to another function. 
      \item What is an easy way to return the last element of an arbitrary vector? 
      \item What does the {\tt paste} function do? 
      \item What is the significance of the ``dot-dot-dot'' argument for a function in R? 
    \end{enumerate}
    \item Write an R function that inputs a vector and computes the mean of the vector. Save your function in an R script called {\tt my\_mean\_func.R}. Be sure to test your function and make sure it is working correctly.
    \item Write an R function that inputs two whole numbers and returns the remainder after dividing the first by the second. Save your function in an R script called {\tt my\_remain\_func.R}. Be sure to test your function and make sure it is working correctly.
    \item Read the first three sections of Chapter 4 Scores and Rankings from \emph{The Data Science Design Manual} (remember that this is available through the library) and answer the following questions.
     \begin{enumerate}
      \item  What is a ``scoring function?''
      \item What is a ``score'' according to the definition given in section 4.2?
      \item Describe an approach or approaches to building effective scoring systems and evaluating a scoring system. 
      \item What is a ranking? Provide some examples. 
      \item What are the characteristics of a good scoring function? 
      \item Describe $Z$-scores and normalization.       
    \end{enumerate}
    \item A (simplified version of a) common scoring function for batters in baseball is the on base percentage (OBP) which is defined as
    \[  OBP = \frac{\text{H} + \text{W}}{\text{AB}}, \]
    where  H is the number of hits for a player, W is the number of walks a player receives, and AB is the number of at bats for a player. 
    \begin{enumerate}
      \item Write a function in R that inputs the number of hits for a player, the number of walks a player receives, and the number of at bats for a player; and then computes the on base percentage. 
      \item Simulate values for H, W, and AB for a season and then use your function to compute OBP for the season. You can use the following code to simulate the values for the season:
      \begin{verbatim}
          AB <- sample(64800:105300,270,replace=TRUE)
          H <- sample(162:16200,270,replace=TRUE)
          W <- sample(162:1782,270,replace=TRUE)
      \end{verbatim}
      What does the {\tt sample} function do? Read the documentation and describe it. 
      \item Use the result of your calculation of OBP based on the simulated  data and report the mean OBP. 
    \end{enumerate}   
    \item Find or make up a formula for some kind of score. Write an R function that implements your formula. Apply your function to some data that is either real or simulated.  Discuss whether your scoring function is good or not.  
    \item Using the {\tt flights} data form the {\tt nycflights13} package,  find all flights that
    \begin{enumerate}
      \item Had an arrival delay of two or more hours
      \item Flew to Houston (IAH or HOU)
      \item Were operated by United, American, or Delta
      \item Departed in summer (July, August, and September)
      \item Arrived more than two hours late, but didn?t leave late
      \item Were delayed by at least an hour, but made up over 30 minutes in flight
      \item Departed between midnight and 6am (inclusive)
    \end{enumerate}
    \item Another useful {\tt dplyr} filtering helper is {\tt between()}. What does it do? Can you use it to simplify the code needed to answer the previous challenges?
    \item How many flights have a missing dep\_time? What other variables are missing? What might these rows represent?
    \item How could you use arrange() to sort all missing values to the start? (Hint: use is.na()).
    \item Sort {\tt flights} to find the most delayed flights. Find the flights that left earliest.
    \item Sort {\tt flights} to find the fastest flights.
  \end{enumerate}
\end{document}
