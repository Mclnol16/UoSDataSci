\documentclass[12pt]{article}
\textwidth=7in
\textheight=9.5in
\topmargin=-1in
\headheight=0in
\headsep=.5in
\hoffset  -.85in

\pagestyle{empty}

\usepackage{amsmath,amssymb,amsfonts}
\usepackage{url}

\renewcommand{\thefootnote}{\fnsymbol{footnote}}
\begin{document}

\begin{center}
{\bf Introduction to Data Science \\ Homework 4: Due Wednesday September 26 at 2:00pm}
\end{center}

\setlength{\unitlength}{1in}

\begin{picture}(6,.1)
\put(0,0) {\line(1,0){6.5}}
\end{picture}

\renewcommand{\arraystretch}{2}

\vskip.25in

\noindent{\bf  {\Large Exercises:} }

\vskip.25in
  \begin{enumerate}
    \item   Go through section 9 Functions of the R Programming course in the {\tt swirl} package, then answer the following questions:
    \begin{enumerate}
      \item     
    \end{enumerate}
    \item Write an R function that inputs a vector and computes the mean of the vector. Save your function in an R script called {\tt my_mean_func.R}.
    \item Find or make up a formula for some kind of score or ranking. Write an R function that implements your formula. Apply your function to some data that is either real or simulated.   
  \end{enumerate}
\end{document}
