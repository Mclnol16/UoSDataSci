\documentclass[12pt]{article}
\textwidth=7in
\textheight=9.5in
\topmargin=-1in
\headheight=0in
\headsep=.5in
\hoffset  -.85in

\pagestyle{empty}

\usepackage{amsmath,amssymb,amsfonts}
\usepackage{url}

\renewcommand{\thefootnote}{\fnsymbol{footnote}}
\begin{document}

\begin{center}
{\bf Introduction to Data Science \\ Homework 5: Due Wednesday October 3 at 2:00pm}
\end{center}

\setlength{\unitlength}{1in}

\begin{picture}(6,.1)
\put(0,0) {\line(1,0){6.5}}
\end{picture}

\renewcommand{\arraystretch}{2}

\vskip.25in

\noindent{\bf  {\Large Exercises:} }

\vskip.25in
  \begin{enumerate}
    \item   Using an appropriate choice of functions in the {\tt dplyr} package, modify the {\tt Salaries.csv} dataset to create a new variable called {\tt hp\_df} that is a data.frame that contains the highest paid player on each team in each year. 
    \item In the {\tt flights} dataset, currently {\tt dep\_time} and {\tt sched\_dep\_time} are convenient to look at, but hard to compute with because they're not really continuous numbers. Convert them to a more convenient representation of number of minutes since midnight. Use {\tt mutate} to add this as a new column. 
    \item What trigonometric functions does R provide? Make plots of each of the trigonometric functions over an appropriate period. 
    \item Brainstorm at least 5 different ways to assess the typical delay characteristics of a group of flights. Consider the following scenarios:
       \begin{itemize}
           \item A flight is 15 minutes early 50\% of the time, and 15 minutes late 50\% of the time.

           \item A flight is always 10 minutes late.

            \item A flight is 30 minutes early 50\% of the time, and 30 minutes late 50\% of the time.

          \item 99\% of the time a flight is on time. 1\% of the time it's 2 hours late.
       \end{itemize}
Which is more important: arrival delay or departure delay? 
\item Read the R Markdown chapter of R for Data Science. (This is chapter 27 of the online version). Use what you learn to create the beginning of a notebook for your semester project. Start with a section titled ``Problem Description and Objectives,'' once you have written it, copy and paste the  background, problem description, and objective into this section. Also include a section titled ``Data Description,'' and copy and paste the data description that you already submitted into this section of your notebook.  
\item Create an R Project and corresponding folder (directory) that contains the notebook you just created, and also a subfolder called Data that contains any data files that you have obtained for your semester project.  
\item {\bf Optional:} If you want to use git or github to maintain your project files (highly recommended), see Dr.\ Graham if you want help with this.   
  \end{enumerate}
\end{document}
